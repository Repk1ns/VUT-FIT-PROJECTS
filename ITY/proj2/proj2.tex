\documentclass[a4paper, 11pt, twocolumn]{article}
\usepackage[text={18cm, 25cm}, left=15mm, top=25mm]{geometry}
\usepackage[czech]{babel}
\usepackage[utf8]{inputenc}
\usepackage[IL2]{fontenc}
\usepackage{times}
\usepackage{amsthm}
\usepackage{amsmath}
\usepackage{amsfonts}

\date{}

\newtheorem{definition}{Definice}
\newtheorem{sentence}{Věta}

\begin{document}
\begin{titlepage}
\begin{center}
{\Huge
\textsc{Fakulta informačních technologií\\[0.4em]Vysoké učení technické v Brně}}\\
\vspace{\stretch{0.382}}
{\Large
Typografie a publikování\,--\,2. projekt\\[0.3em]
Sazba dokumentů a matematických výrazů
\vspace{\stretch{0.618}}
}
\end{center}

{\Large 2021 \hfill Vojtěch Mimochodek (xmimoc01)}
\end{titlepage}

\section*{Úvod}
V~této úloze si vyzkoušíme sazbu titulní strany, matematic- kých vzorců, prostředí a~dalších textových struktur obvyklých pro technicky zaměřené texty (například rovnice (\ref{Equ1}) nebo Definice \ref{Def1} na straně \pageref{Def1}). Rovněž si vyzkoušíme používání odkazů \verb|\refa| a~\verb|\pageref|. 

Na titulní straně je využito sázení nadpisu podle optického středu s~využitím zlatého řezu. Tento postup byl probírán na přednášce. Dále je použito odřádkování se zadanou relativní velikostí 0.4 em a 0.3 em. 

V~případě, že budete potřebovat vyjádřit matematickou konstrukci nebo symbol a~nebude se Vám dařit jej nalézt v~samotném {\LaTeX}u, doporučuji prostudovat možnosti balíku maker {\AmS}-{\LaTeX}.

\section{Matematický text}
Nejprve se podíváme na sázení matematických symbolů a~výrazů v~plynulém textu včetně sazby definic a~vět s~využitím balíku \texttt{amsthm}. Rovněž použijeme poznámku pod čarou s~použitím příkazu \verb|\footnote|. Někdy je vhodné použít konstrukci \verb|\mbox{}|, která říká, že text nemá být zalomen.

\begin{definition}\label{Def1}
\textnormal{Rozšířený zásobníkový automat} (RZA) je definován jako sedmice tvaru $A = (Q, \Sigma, \Gamma, \delta, q_0, Z_0, F)$, kde:
\end{definition}

\begin{itemize}
\item \textit{$Q$ je konečná množina} vnitřních (řídicích) stavů,
\item \textit{$\Sigma$ je konečná} vstupní abeceda,
\item \textit{$\Gamma$ je konečná} zásobníková abeceda,
\item \textit{$\delta$ je} přechodová funkce $Q\times(\Sigma\cup\{\epsilon\})\times\Gamma^{*}\rightarrow2^{Q\times\Gamma^{*}}$,
\item $q_0 \in Q$ \textit{je} počáteční stav, $Z_0 \in \Gamma$ \textit{je} startovací symbol zásobníku \textit{a $F \subseteq Q$ je množina} koncových stavů.
\end{itemize}

Nechť $P = (Q, \Sigma, \Gamma, \delta, q_0, Z_0, F)$ je rozšířený zásobníkový automat. \textit{Konfigurací} nazveme trojici $(q, w, \alpha) \in Q\times\Sigma^{*}\times\Gamma^{*}$, kde $q$ je aktuální stav vnitřního řízení, $w$ je dosud nezpracovaná část vstupního řetězce a~$\alpha = Z_{i_1}Z_{i_2}\ldots Z_{i_k}$ je obsah zásobníku\footnote{$Z_{i_1}$ je vrchol zásobníku}.

\subsection{Podsekce obsahující větu a odkaz}

\begin{definition}\label{Def2}
\textnormal{Řetězec $w$ nad abecedou $\Sigma$ je přijat RZA} A~jestliže $(q_0, w, Z_0)\underset{A}{\overset{*}{\vdash}}(q_F, \epsilon, \gamma)$ pro nějaké $\gamma \in \Gamma^{*}$ a $q_F \in F$. Množinu $L(A) = \{w \ \mid \ w \textup{ je přijat RZA }A \}\subseteq \Sigma^{*}$ nazýváme \textnormal{jazyk přijímaný RZA} A.
\end{definition}

Nyní si vyzkoušíme sazbu vět a~důkazů opět s~použitím balíku \texttt{amsthm}.

\begin{sentence}
Třída jazyků, které jsou přijímány ZA, odpovídá \textnormal{bezkontextovým jazykům}.
\end{sentence}

\begin{proof}
\textnormal{V~důkaze vyjdeme z~\ref{Def1} a~\ref{Def2}.}
\end{proof}

\section{Rovnice a odkazy}

Složitější matematické formulace sázíme mimo plynulý text. Lze umístit několik výrazů na jeden řádek, ale pak je třeba tyto vhodně oddělit, například příkazem \verb|\quad|.
$$\sqrt[i]{x_i^{3}}\quad \text{kde $x_i$ je $i$-té sudé číslo splňující}\quad x_i^{x_i^{i^{2}+2}}\leq y_i^{x_i^{4}}$$

V~rovnici (\ref{Equ1}) jsou využity tři typy závorek s~různou explicitně definovanou velikostí.

\begin{eqnarray}\label{Equ1}
x & = & \bigg[ \Big\{ \big[a+b\big]\ast c\Big\}^d \oplus 2\bigg]^{3/2}\\
y & = & \lim\limits_{x \to \infty} \frac{\frac{1}{\log_{10} x}}
{\textnormal{\fontfamily{cmr}\selectfont{sin}}^2 x + \textnormal{\fontfamily{cmr}\selectfont{cos}}^2 x}\nonumber
\end{eqnarray}

V~této větě vidíme, jak vypadá implicitní vysázení limity $\lim_{n\rightarrow \infty}f(n)$ v~normálním odstavci textu. Podobně je to i~s~dalšími symboly jako $\prod _{i=1}^n 2^i$ či $\bigcap_{A\in\mathcal{B}}A$. V~případě vzorců $\lim\limits_{n\to \infty}f(n)$ a $\prod\limits _{i=1}^n 2^i$ jsme si vynutili méně úspornou sazbu příkazem \verb|\limits|.

\begin{eqnarray}\label{Equ2}
\int _b^a g(x)dx = \minus \int\limits_{a}^b f(x)dx
\end{eqnarray}

\section{Matice}

Pro sázení matic se velmi často používá prostředí \texttt{array} a~závorky (\verb|\left|, \verb|\right|).

$$\left(
\begin{array}{ccc}
a-b & \widehat{\xi + \omega} & \pi\\
\vec{\mathbf{a}} & \overleftrightarrow{AC} & \widehat{\beta}
\end{array}
\right) = 1 \Longleftrightarrow \mathcal{Q} = \mathbb{R}$$

$$A = \left\|
\begin{array}{cccc}
a_{11} & a_{12} & \ldots & a_{1n}\\
a_{21} & a_{22} & \ldots & a_{2n}\\
\vdots & \vdots & \ddots & \vdots\\
a_{m1} & a_{m2} & \ldots & a_{mn}
\end{array}
\right\| = \left|
\begin{array}{cc}
t & u\\
v & w
\end{array}
\right| = tw \minus uv$$

Prostředí \texttt{array} lze úspěšně využít i~jinde.

$$\binom{n}{k} =
\left\{
\begin{array}{cc}
0 & \mbox{pro $k < 0 \text{ nebo } k > n$}\\
\frac{n!}{k!(n\minus k)!} & \mbox{pro $0 \leq k \leq n$}.
\end{array}
\right.$$

\end{document}